\documentclass[12pt,letterpaper,twoside]{article}
\usepackage[utf8]{inputenc}
\usepackage[francais]{babel}
\usepackage[T1]{fontenc}
\usepackage{fullpage}
\usepackage{amsmath}
\usepackage{amsfonts}
\usepackage{amssymb}
\usepackage{pdfpages}
\usepackage{setspace}
\usepackage{float}
\usepackage{hyperref}
\usepackage{color}
\usepackage{multirow}
\usepackage{tabularx}

\onehalfspacing
\begin{document}

\setcounter{secnumdepth}{0}
\begin{titlepage}

        \vspace*{1cm}
        \begin{small}
        \begin{tabularx}{\textwidth}{ l X r }
        \multirow{3}{*}{\includegraphics[height=1.5cm,keepaspectratio]{ul_logo.pdf}}
        && \'Equipe IA\\
        && Projet Robocup\\
        && Été 2016\\

        \scriptsize{\textbf{FACULTÉ DES SCIENCES ET GÉNIE}} && Robocup
        \end{tabularx}
        \end{small}

        \vfill

        \begin{center}

        Gestion de projet Robocup

        \vspace{0.5cm}

        Rencontre d'\'equipe

        \vspace{2cm}

        \end{center}

        \vfill

        Date: 28 avril 2016

        \vspace{0.4cm}

        \rule{\textwidth}{2pt}

        \vspace{0.3cm}

        \begin{tabularx}{\textwidth}{ l X r }

        \textbf{Robocup} && \textbf{\'Equipe IA} \\

        \end{tabularx}


\end{titlepage}


\section*{Proc\`es verbal de la réunion du 28 avril 2016}

\subsection*{Fin du mandat de Julien Becirovski}
Le mandat de Julien Becirovski en tant que coordonateur de l'équipe IA vient de se terminer.
Il a décidé de ne pas renouveler, et ce sera Alexandre Gingras-Courchesne qui prendra sa place.

\subsection*{Centralisation de la gestion sur GitHub}
Le Trello de Robocup ainsi que le Google Drive seront abandonnés.
La gestion des t\^aches sera maintenant faite enti\`erement sur le GitHub. 

\subsection*{Planification de l'été 2016}

\subsubsection*{Estimation du temps}
La planification de l'été a pris en considération que chaque membre de l'équipe IA de Robocup mettra en moyenne environ 5 à 9 heures par semaine sur le projet.

\subsubsection*{Milestones}
L'objectif pour l'été est de pouvoir simuler des compétitions 3 versus 3.
Le résultat sera présenté au directeur du département IFT-GLO.

\begin{enumerate}
\item Récupération des primitives de la compétition et intégration dans la STA.
\item Implémentaion de stratégies
\item Présentation et tests
\end{enumerate}

\subsection*{T\^aches du premier sprint}

\begin{itemize}
\item Modification de l'engin
\item Implémentation de STA 2.0
\item Centralisation de la gestion sur GitHub
\item Documentation
\item Pont haut niveau/bas niveau
\item Tools pour le bas niveau/ bench automatique
\end{itemize}

Une énumération plus détaillée des tâches sera mise sur le GitHub de Robocup ULaval.

\subsection*{Membres présents}
\begin{itemize}
\item Julien B.
\item Ryma H.
\item David C.
\item Alexandre G.
\item Félix P.
\item Julien M.
\item Alexandra M.
\item Maxime G.
\item Yassine Z.
\item Maxime M.
\end{itemize}

\end{document}
