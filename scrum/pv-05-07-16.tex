\documentclass[12pt,letterpaper,twoside]{article}
\usepackage[utf8]{inputenc}
\usepackage[francais]{babel}
\usepackage[T1]{fontenc}
\usepackage{fullpage}
\usepackage{amsmath}
\usepackage{amsfonts}
\usepackage{amssymb}
\usepackage{pdfpages}
\usepackage{setspace}
\usepackage{float}
\usepackage{hyperref}
\usepackage{color}
\usepackage{multirow}
\usepackage{tabularx}

\onehalfspacing
\begin{document}

\setcounter{secnumdepth}{0}
\begin{titlepage}

        \vspace*{1cm}
        \begin{small}
        \begin{tabularx}{\textwidth}{ l X r }
        \multirow{3}{*}{\includegraphics[height=1.5cm,keepaspectratio]{ul_logo.pdf}}
        && \'Equipe IA\\
        && Projet Robocup\\
        && Été 2016\\

        \scriptsize{\textbf{FACULTÉ DES SCIENCES ET GÉNIE}} && Robocup
        \end{tabularx}
        \end{small}

        \vfill

        \begin{center}

        Gestion de projet Robocup

        \vspace{0.5cm}

        Rencontre d'\'equipe

        \vspace{2cm}

        \end{center}

        \vfill

        Date: 07 mai 2016

        \vspace{0.4cm}

        \rule{\textwidth}{2pt}

        \vspace{0.3cm}

        \begin{tabularx}{\textwidth}{ l X r }

        \textbf{Robocup} && \textbf{\'Equipe IA} \\

        \end{tabularx}


\end{titlepage}


\section*{Proc\`es-verbal de la réunion du 07 mai 2016}


\subsection{Division des tâches et assignation}
\begin{itemize}
	\item Compilation windows : Alexandre G.
	\item Documentation wiki : David C, Félix P, Alexandra M, Alexandre G, Maxime M.
	\item Clean-up du code : Félix P, Alexandre G, Maxime M.
	\item Créer un how-to pour une stratégie de base : ???
\end{itemize}
Tâches pour tous : Lire le papier sur STA (STP), faire des unit tests sur son code, utiliser le standard PEP-8.

\subsection*{Retour avancement}
\begin{itemize}
	\item Le plan initial de synchronisation de temps avec l'engin physique en utilisant les timestamp de GrSim n'a pas pu être implémenté pour des raisons techniques. Félix a résolu le problème de façon temporaire en modifiant le code source de GrSim.
	\item Du côté de la consommation de CPU de GrSim, Félix a proposé d'ajouter une option permettant de diminuer la consommation en cas de besoin. Il se chargera de l'implémenter.
	\item La documentation à été commencée par Alexandre G, plusieurs membres de l'équipe se chargeront de l'améliorer.
	\item Les dépôts GitHub ont été restructuré dans une organisation (RobocupULavalHautNiveau). 
\end{itemize}

\subsection*{Utilisation de git pour versionner le code source}
La documentation du projet sera maintenant disponible sur le dépôt "Admin".
Avant d'être intégré à la branche dev, le code devra être révisé par Alexandre G, Félix P ou Maxime M.
Tout bug ou travail devra faire l'objet d'une issue.

\subsection*{Rappel slack pour les communications}
Alexandre G. a rappelé à l'équipe que les communications se feront par Slack, autant pour les informations concernant les réunions que pour les questionnements plus techniques.
De plus, il serait apprécié que les membres de l'équipe réagissent de temps à autre aux discussions.

\subsection{Unit test -- int\'egration continue}
L'objectif de base est que, pour tout nouveau code, des tests unitaires soient développés simultanément.
Ajouter des tests pour le code actuel est une des tâches majeures du milestone 1.
La couverture souhaitée est de 90\%.
Suite à l'implémentation des tests, un système d'intégration continue avec nightly build sera mis en place.

\subsection{Affichage debug}
Un affichage de debug sera réalisé par Julien B. 
L'affichage devra être utilisable par l'équipe ia et par les équipes bas-niveau/électrique.

\subsection{team-intégration}
Suite à la proposition de Philippe Turgeon, une équipe d'intégration sera mise en place pour travailler directement sur le robot. 
Les rencontres de cette équipe seront, en général, le mercredi, et la présence de quelques membres de la team ia est requise.
Le but est d'encourager le travail multi-disciplinaire.


\subsection*{Membres présents}
\begin{itemize}
\item Julien B.
\item Ryma H.
\item David C.
\item Alexandre G.
\item Félix P.
\item Étienne B.
\item Alexandra M.
\item Maxime G.
\item Yassine Z.
\item Maxime M.
\end{itemize}

\end{document}
