\documentclass[12pt,letterpaper,twoside]{article}
\usepackage[utf8]{inputenc}
\usepackage[francais]{babel}
\usepackage[T1]{fontenc}
\usepackage{fullpage}
\usepackage{amsmath}
\usepackage{amsfonts}
\usepackage{amssymb}
\usepackage{pdfpages}
\usepackage{setspace}
\usepackage{float}
\usepackage{hyperref}
\usepackage{color}
\usepackage{multirow}
\usepackage{tabularx}

\onehalfspacing
\begin{document}

\setcounter{secnumdepth}{0}
\begin{titlepage}

        \vspace*{1cm}
        \begin{small}
        \begin{tabularx}{\textwidth}{ l X r }
        \multirow{3}{*}{\includegraphics[height=1.5cm,keepaspectratio]{ul_logo.pdf}}
        && \'Equipe IA\\
        && Projet Robocup\\
        && Été 2016\\

        \scriptsize{\textbf{FACULTÉ DES SCIENCES ET GÉNIE}} && Robocup
        \end{tabularx}
        \end{small}

        \vfill

        \begin{center}

        Gestion de projet Robocup

        \vspace{0.5cm}

        Rencontre d'\'equipe

        \vspace{2cm}

        \end{center}

        \vfill

        Date: 14 mai 2016

        \vspace{0.4cm}

        \rule{\textwidth}{2pt}

        \vspace{0.3cm}

        \begin{tabularx}{\textwidth}{ l X r }

        \textbf{Robocup} && \textbf{\'Equipe IA} \\

        \end{tabularx}


\end{titlepage}


\section*{Proc\`es-verbal de la réunion du 14 mai 2016}


\subsection{Tâches à faire}
\begin{itemize}
	\item Mettre notre code sous une licence : MIT ou GNU.
	\item Lire et avancer la documentation.
	\item Lire le code et le documenter selon les normes PEP-8.
	\item Inspecter le code concernant le Behavior tree.
	\item À tout le monde: Penser commandite!
\end{itemize}

\subsection*{Retour avancement}
\begin{itemize}
	\item Le tutoriel pour la caméra est en cours.
	\item L'archivage de l'ordinateur d'intégration a été fait.
	\item Le Pathfinder est toujours en avancement.
\end{itemize}


\subsection*{Membres présents}
\begin{itemize}
\item David C.
\item Alexandre G.
\item Félix P.
\item Étienne B.
\item Maxime G.
\item Yassine Z.
\end{itemize}

\end{document}
