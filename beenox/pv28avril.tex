\documentclass[12pt,letterpaper,twoside]{article}
\usepackage[utf8]{inputenc}
\usepackage[francais]{babel}
\usepackage[T1]{fontenc}
\usepackage{fullpage}
\usepackage{amsmath}
\usepackage{amsfonts}
\usepackage{amssymb}
\usepackage{pdfpages}
\usepackage{setspace}
\usepackage{float}
\usepackage{hyperref}
\usepackage{color}
\usepackage{multirow}
\usepackage{tabularx}

\onehalfspacing
\begin{document}

\setcounter{secnumdepth}{0}
\begin{titlepage}

        \vspace*{1cm}
        \begin{small}
        \begin{tabularx}{\textwidth}{ l X r }
        \multirow{3}{*}{\includegraphics[height=1.5cm,keepaspectratio]{ul_logo.pdf}}
        && \'Equipe IA\\
        && Projet Robocup\\
        && Été 2016\\

        \scriptsize{\textbf{FACULTÉ DES SCIENCES ET GÉNIE}} && Robocup
        \end{tabularx}
        \end{small}

        \vfill

        \begin{center}

        Gestion de projet Robocup

        \vspace{0.5cm}

        Rencontre d'\'equipe

        \vspace{2cm}

        \end{center}

        \vfill

        Date: 28 avril 2016

        \vspace{0.4cm}

        \rule{\textwidth}{2pt}

        \vspace{0.3cm}

        \begin{tabularx}{\textwidth}{ l X r }

        \textbf{Robocup} && \textbf{\'Equipe IA} \\

        \end{tabularx}


\end{titlepage}


\section*{Rencontre avec Beenox, 29 avril 2016}

\subsection*{Organisation du premier sprint de l'été 2016}
\subsection*{Probl\`emes de performances avec l'engin de compétition}
\subsection*{Abstraction du niveau des startégies}
\subsection*{Intégration des nouveaux}
\subsection*{Références}
Alexandre Bergeron nous a envoyé une liste de livres et de sites que l'équipe pourra utiliser comme référence sur l'intelligence artificielle.
\subsection*{Debug dans les jeux vidéos}

\subsection*{Autres conseils}

\subsection*{Membres présents}
\begin{itemize}
\item Ryma H.
\item Alexandre G.
\item Alexandra M.
\item Maxime G.
\item Maxime M.
\end{itemize}

\end{document}
