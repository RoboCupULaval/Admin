\documentclass[12pt,letterpaper,twoside]{article}
\usepackage[utf8]{inputenc}
\usepackage[francais]{babel}
\usepackage[T1]{fontenc}
\usepackage{fullpage}
\usepackage{amsmath}
\usepackage{amsfonts}
\usepackage{amssymb}
\usepackage{pdfpages}
\usepackage{setspace}
\usepackage{float}
\usepackage{hyperref}
\usepackage{color}
\usepackage{multirow}
\usepackage{tabularx}

\onehalfspacing
\begin{document}

\setcounter{secnumdepth}{0}
\begin{titlepage}

        \vspace*{1cm}
        \begin{small}
        \begin{tabularx}{\textwidth}{ l X r }
        \multirow{3}{*}{\includegraphics[height=1.5cm,keepaspectratio]{ul_logo.pdf}}
        && \'Equipe IA\\
        && Projet Robocup\\
        && Été 2016\\

        \scriptsize{\textbf{FACULTÉ DES SCIENCES ET GÉNIE}} && Robocup
        \end{tabularx}
        \end{small}

        \vfill

        \begin{center}

        Gestion de projet Robocup

        \vspace{0.5cm}

        Rencontre d'\'equipe

        \vspace{2cm}

        \end{center}

        \vfill

        Date: 28 avril 2016

        \vspace{0.4cm}

        \rule{\textwidth}{2pt}

        \vspace{0.3cm}

        \begin{tabularx}{\textwidth}{ l X r }

        \textbf{Robocup} && \textbf{\'Equipe IA} \\

        \end{tabularx}


\end{titlepage}


\section*{Rencontre avec Beenox, 29 avril 2016}

\subsection*{Organisation du premier sprint de l'été 2016}
Les membres de Beenox nous ont donné quelques recommendations pour le sprint de cet été. 
\begin{itemize}
\item \textbf{Temps de sprint maximal: 2 ou 3 semaines}
\item \textbf{Trouver un client et définir une date de fin précise} 
Plut\^ot que de définir une date arbitraire, trouver un <<client>>, par exemple un professeur, `a qui l'on pourrait montrer notre progr\`es régulièrement.
\item \textbf{Si la tâche n'est pas écrite, elle ne sera pas faite.}
\item \textbf{Avoir à la fin de chaque sprint un <<sprint review>>.}
On peut ainsi mieux évaluer nos estiamtions de temps pour le prochain sprint.
\end{itemize}
Les membres de Beenox nous ont aussi proposé de mettre les tâches de notre sprint sur Slack, pour qu'ils puissent l'évaluer.

\subsection*{Probl\`emes de performances avec l'engin de compétition}

\subsection*{Abstraction du niveau des startégies}

\subsection*{Intégration des nouveaux}
Pour faciliter l'intégration des nouveaux, il faudrait avoir:
\begin{itemize}
\item Un wiki GitHub pour la documentation
\item Un << How to >>
\begin{itemize}
\item Un guide d'installation. 
Il faut que n'importe qui puisse suivre les instructions et avoir une simulation de jeux.
\item Identifier sur chaque guide la/les personne/s resssource/s \textbf{avec une photo}. 
\end{itemize}
\end{itemize}

\subsection*{Références}
Alexandre Bergeron nous a envoyé une liste de livres et de sites que l'équipe pourra utiliser comme référence sur l'intelligence artificielle.

\subsection*{Debug dans les jeux vidéos}

\subsection*{Autres conseils}

\subsection*{Membres présents}
\begin{itemize}
\item Ryma H.
\item Alexandre G.
\item Alexandra M.
\item Maxime G.
\item Maxime M.
\end{itemize}

\end{document}
